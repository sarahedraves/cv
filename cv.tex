%-----------------------------------------------------------------------------------------------------------------------------------------------%
%	The MIT License (MIT)
%
%	Copyright (c) 2021 Jitin Nair
%
%	Permission is hereby granted, free of charge, to any person obtaining a copy
%	of this software and associated documentation files (the "Software"), to deal
%	in the Software without restriction, including without limitation the rights
%	to use, copy, modify, merge, publish, distribute, sublicense, and/or sell
%	copies of the Software, and to permit persons to whom the Software is
%	furnished to do so, subject to the following conditions:
%	
%	THE SOFTWARE IS PROVIDED "AS IS", WITHOUT WARRANTY OF ANY KIND, EXPRESS OR
%	IMPLIED, INCLUDING BUT NOT LIMITED TO THE WARRANTIES OF MERCHANTABILITY,
%	FITNESS FOR A PARTICULAR PURPOSE AND NONINFRINGEMENT. IN NO EVENT SHALL THE
%	AUTHORS OR COPYRIGHT HOLDERS BE LIABLE FOR ANY CLAIM, DAMAGES OR OTHER
%	LIABILITY, WHETHER IN AN ACTION OF CONTRACT, TORT OR OTHERWISE, ARISING FROM,
%	OUT OF OR IN CONNECTION WITH THE SOFTWARE OR THE USE OR OTHER DEALINGS IN
%	THE SOFTWARE.
%	
%
%-----------------------------------------------------------------------------------------------------------------------------------------------%

%----------------------------------------------------------------------------------------
%	DOCUMENT DEFINITION
%----------------------------------------------------------------------------------------

% article class because we want to fully customize the page and not use a cv template
\documentclass[a4paper,12pt]{article}

%----------------------------------------------------------------------------------------
%	FONT
%----------------------------------------------------------------------------------------

% % fontspec allows you to use TTF/OTF fonts directly
% \usepackage{fontspec}
% \defaultfontfeatures{Ligatures=TeX}

% % modified for ShareLaTeX use
% \setmainfont[
% SmallCapsFont = Fontin-SmallCaps.otf,
% BoldFont = Fontin-Bold.otf,
% ItalicFont = Fontin-Italic.otf
% ]
% {Fontin.otf}

%----------------------------------------------------------------------------------------
%	PACKAGES
%----------------------------------------------------------------------------------------
\usepackage{url}
\usepackage{parskip} 	

%other packages for formatting
\RequirePackage{color}
\RequirePackage{graphicx}
\usepackage[usenames,dvipsnames]{xcolor}
\usepackage[scale=0.9]{geometry}

%tabularx environment
\usepackage{tabularx}

%for lists within experience section
\usepackage{enumitem}

% centered version of 'X' col. type
\newcolumntype{C}{>{\centering\arraybackslash}X} 

%to prevent spillover of tabular into next pages
\usepackage{supertabular}
\usepackage{tabularx}
\newlength{\fullcollw}
\setlength{\fullcollw}{0.47\textwidth}

%custom \section
\usepackage{titlesec}				
\usepackage{multicol}
\usepackage{multirow}

%CV Sections inspired by: 
%http://stefano.italians.nl/archives/26
\titleformat{\section}{\Large\scshape\raggedright}{}{0em}{}[\titlerule]
\titlespacing{\section}{0pt}{10pt}{10pt}

%for publications
\usepackage[style=authoryear,sorting=ynt, maxbibnames=2]{biblatex}

%Setup hyperref package, and colours for links
\usepackage[unicode, draft=false]{hyperref}
\definecolor{linkcolour}{rgb}{0,0.2,0.6}
\hypersetup{colorlinks,breaklinks,urlcolor=linkcolour,linkcolor=linkcolour}
\addbibresource{citations.bib}
\setlength\bibitemsep{1em}

%for social icons
\usepackage{fontawesome5}

%debug page outer frames
%\usepackage{showframe}

\usepackage[none]{hyphenat} % in your preamble (disables hyphenation globally)
\usepackage{array}          % needed for \arraybackslash inside tabularx

% job listing environments

\newenvironment{jobshort}[2]
    {
    \begin{tabularx}{\linewidth}{@{}X r@{}}
    {\raggedright\arraybackslash\nohyphens{#1}} & #2 \\[3.75pt]
    \end{tabularx}
    }
    {}



\newenvironment{joblong}[2]
    {
    \begin{tabularx}{\linewidth}{@{}l X r@{}}
    #1 & \hfill &  #2 \\[3.75pt]
    \end{tabularx}
    \begin{minipage}[t]{\linewidth}
    \begin{itemize}[nosep,after=\strut, leftmargin=1em, itemsep=3pt,label=--]
    }
    {
    \end{itemize}
    \end{minipage}    
    }



%----------------------------------------------------------------------------------------
%	BEGIN DOCUMENT
%----------------------------------------------------------------------------------------
\begin{document}

% non-numbered pages
\pagestyle{empty} 

%----------------------------------------------------------------------------------------
%	TITLE
%----------------------------------------------------------------------------------------

% \begin{tabularx}{\linewidth}{ @{}X X@{} }
% \huge{Your Name}\vspace{2pt} & \hfill \emoji{incoming-envelope} email@email.com \\
% \raisebox{-0.05\height}\faGithub\ username \ | \
% \raisebox{-0.00\height}\faLinkedin\ username \ | \ \raisebox{-0.05\height}\faGlobe \ mysite.com  & \hfill \emoji{calling} number
% \end{tabularx}

\begin{tabularx}{\linewidth}{@{} C @{}}
\Huge{Sarah E. Draves} \\[7.5pt]
\href{https://github.com/sarahedraves}{\raisebox{-0.05\height}\faGithub\ sarahedraves} \ $|$ \ 
\href{https://linkedin.com/in/sarahedraves}{\raisebox{-0.05\height}\faLinkedin\ sarahedraves} \ $|$ \ 
\href{https://sarahedraves.github.io/}{\raisebox{-0.05\height}\faGlobe \ website} \ $|$ \ 
\href{mailto:sarah@draves.org}{\raisebox{-0.05\height}\faEnvelope \ sarah@draves.org} \ $|$ \ 
\href{tel:+12066696004}{\raisebox{-0.05\height}\faMobile \ +1.206.669.6004} \\
\end{tabularx}

%Interests/ Keywords/ Summary
\section{Summary}
Astrophysics researcher and physics undergraduate with a strong foundation in computational methods, data analysis, and physics coursework. Experienced in working with large astronomical datasets with various Python packages. Currently applying to physics and astronomy PhD programs with the goal of continuing research in observational astrophysics.

%----------------------------------------------------------------------------------------
%	EDUCATION
%----------------------------------------------------------------------------------------

%Education
\section{Education}

\begin{joblong}{\textbf{CUNY New York City College of Technology}, Bachelor of Science}{Jan 2024 -- Dec 2025}
\item Major: Applied Computational Physics, Overall GPA: 4.00/4.00
\item Relevant Coursework: Statistical Physics and Thermodynamics, Electricity and Magnetism, Classical Mechanics, Quantum Mechanics, Modern Physics, Machine Learning for Physics, Partial Differential Equations, Computational Astrophysics (CUNY Graduate Center)
\end{joblong}

\begin{joblong}{\textbf{CUNY Borough of Manhattan Community College}}{Jan -- Dec 2023}
\item Overall GPA: 4.00/4.00
\item Relevant coursework: General Astronomy, University Physics I, Ordinary Differential Equations
\end{joblong}

\begin{joblong}{\textbf{Colgate University}, Bachelor of Arts}{Sep 2016 -- Dec 2019}
\item Major: Mathematical Economics, Major GPA: 3.90/4.00
\item Graduated \textit{magna cum laude} one semester early
\item Dean's List with Distinction (4 semesters); Dean's List (2 semesters)
\item Study abroad: Colgate London Economics Study Group, Fall 2018
\item Relevant Coursework: Computational Mathematics, Computer Science II, Computer Organization, Linear Algebra, Multivariable Calculus, Probability
\end{joblong}

%----------------------------------------------------------------------------------------
% EXPERIENCE SECTIONS
%----------------------------------------------------------------------------------------

%Experience
\section{Research Experience}

\begin{joblong}{\textbf{AstroCom NYC}}{Jan 2025 -- Present}
\item Undergraduate research fellowship for students at CUNY schools interested in pursuing graduate studies in Astronomy
\item Included the class Scientific Thought \& Practices, which emphasizes experimental design and the scientific method
\item Funded by the National Science Foundation
\item Also serving as CUNY Astro Student Union Representative in Fall 2025, which involves liasing between undergraduates, graduate students, and faculty as well as planning events
\end{joblong}

\begin{joblong}{\textbf{Brown Dwarfs in New York City}}{May 2025 -- Present}
\item Research assistant to Dr. Mark Popinchalk, a post-doc at the American Museum of Natural History, as part of the AstroCom NYC program
\item Compiled a catalog of Complex Rotator M Dwarf Stars from past papers and analyzed them with data from TESS, GAIA, and K2 to learn more about their characteristics and changes over time
\item Targeting paper submission in October 2025
\item The BDNYC Group is run by Dr. Jackie Faherty (AMNH) and Dr. Kelle Cruz (CUNY)
\end{joblong}

\begin{joblong}{\textbf{Gotham Web Lab}}{May 2024 -- Present}
\item Research assistant to Dr. Charlotte Welker (CUNY faculty) and Dr. Charlotte Olsen (CUNY post-doc)
\item Using Legacy Survey of Space and Time simulated data to develop and improve identification method of 2D cosmic filaments, taking into account the uncertainties of photometric redshifts
\end{joblong}

\begin{joblong}{\textbf{Emerging Scholars Program}, CUNY New York City College of Technology}{Feb -- May 2024}
\item Completed research project on Galaxy Bulge Regions using Sloan Digital Sky Survey data under the guidance of Dr. Ari Maller 
\end{joblong}

\section{Presentations and Posters}
\begin{jobshort}{\textbf{A Library of Light Curves: A Catalogue of All Known Complex Rotators through K2 and TESS}}{7.30.2025}
Presented at the 2025 Undergraduate Summer Research Symposium at the American Museum of Natural History in New York City. Abstract: Complex rotators (CRs) are a recently discovered type of young M dwarf star that have unusual light curves, showing periodic behavior that cannot be explained by more typical causes like starspots. Previous studies have focused on identifying CRs and have suffered from a lack of observations over long temporal baselines. In this project, we compiled a data set of previously identified CRs in the literature and their available K2 and TESS light curves. Also using Gaia data, we investigate the similarities between these objects as well as how they change over time. In the multiyear timeframe we looked at (in some cases over a decade), we found CRs changed their complex patterns significantly or lost their complexity entirely. Further analysis of the light curves could yield better insights into the causes of these changes. This data set will be pivotal for others in the field to continue studying this unusual phenomenon.
\end{jobshort}

\begin{jobshort}{\textbf{Reconstructing Cosmic Filaments around Dwarf Galaxies with the Rubin Observatory}}{1.14.2025 \& 12.4.2024}
Presented similar posters at the 245th meeting of the American Astronomical Society in National Harbor, Maryland and at the 41st Semi-Annual Dr. Janet Liou-Mark Honors \& Undergraduate Research Poster Presentation at CUNY New York City College of Technology. \href{https://aas245-aas.ipostersessions.com/Default.aspx?s=3B-03-9F-6A-6E-75-01-97-E0-FC-B1-F5-ED-BF-AB-40}{AAS poster link} and \href{https://academicworks.cuny.edu/ny_pubs/1222/}{CUNY poster link}. Abstract: While the largest structures in the universe that make up the cosmic web are known to influence galaxy formation for massive galaxies, the details of which processes and conditions drive the growth of low mass galaxies remain unclear. Dwarf galaxies are particularly sensitive to the environment around them and are thus an ideal way to trace smaller filaments and then investigate how exactly they impact galactic evolution. To accomplish this, we use simulated observations for the upcoming Legacy Survey of Space and Time (LSST) conducted by the Rubin Observatory, which is expected to discover 20 billion new galaxies with excellent completeness for dwarfs in the nearby universe. We use this data to identify and select low mass galaxies and infer their properties, such as position, redshift stellar mass, and star formation rate. We bin these galaxies by redshift and use DisPerSE, a topological algorithm which we use recursively in a tomographic way, to reconstruct two-dimensional filaments. We then investigate how the properties of the dwarf galaxies are correlated with their proximity to a filament. This pipeline will be applied to upcoming LSST data, where the unprecedented wealth of data promises to unlock deep insights into the details of how the cosmic web drives galaxy evolution.
\end{jobshort}

\begin{jobshort}{\textbf{Examining Galaxy Bulge Regions with the Sloan Digital Sky Survey}}{5.9.2024}
Presented at the 40th Semi-Annual Dr. Janet Liou-Mark Honors \& Undergraduate Research Poster Presentation at CUNY New York City College of Technology. \href{https://academicworks.cuny.edu/ny_pubs/1137/}{Poster link}. Abstract: Nearly all of the ordinary matter in the universe is located in galaxies, which are made up stars, gas, dust, and black holes, and range in size from a few thousand to a few hundred thousand light years across. Galaxies come in different shapes, but many of them are spiral shaped, and some of those have a central bulge region that is distinct from the rest of the galactic disk. This project used a large data set of galaxies that had their bulge and disk components separated to see what correlations those fractions of each galaxy have with other galaxy properties, including total stellar mass and redshift. Binned statistics techniques were used to reduce the noise in the data. The relationships between the bulge radius and total mass and the disk radius and total mass showed strong linear correlations. The relationship between bulge to total ratio and total mass showed that the bulge fraction increases with mass dramatically at first and then declines slowly with mass, for both the mass ratio and the luminosity ratio. This information can be used to constrain galaxy formation models to learn more about how the universe came to be as it is today.
\end{jobshort}
  
%Relevant Work Experience
\section{Relevant Work Experience}

\begin{joblong}{\textbf{ibidPrep}, Associate Tutor}{Oct 2023 -- Present}
\item Tutoring middle school and high school students for test prep, math, high school physics, and general academic skills
\end{joblong}

\begin{joblong}{\textbf{Highspot}, Data Analyst}{Apr 2020 -- Jun 2023}
\item Managed analytics projects from initial conception through finished products, documentation, and maintenance
\item Designed dashboards using Tableau Desktop for both Sales and Marketing teams, ranging from individual requests for specific analysis to dashboards that have been in use for years and used by dozens of people
\item Created multiple data sources using SQL within Snowflake that combined data from Salesforce, Workday, and Google Sheets to be used for analysis of different areas of the business
\item Performed regular data validation of key Tableau Online dashboards against Salesforce data to ensure that customer teams always had access to correct information and confidence in the data they used to make business decisions
\item Wrote and maintained documentation of popular dashboards so that new users could train themselves on the dashboards and to reduce the number of clarification questions directed to the analytics team
\item Completed service request tickets, including updating calculated fields and data sources to match changes to the Salesforce schema, new feature requests for existing dashboards, and one-off analysis projects, to support the needs of our customer teams and efficiently respond to feedback
\item Led meetings regularly with different customer teams and go-to-market executives to share data-driven insights, discuss possible courses of action, gather feedback on analytics projects, and stay up-to-date on what other teams were working on
\end{joblong}

\begin{joblong}{\textbf{Colgate Economics Department}, Intermediate Microeconomics Tutor}{Sep 2017 -- May 2018}
\item Held weekly tutoring sessions to help students practice problems, understand concepts, and prepare for exams
\item Graded problem sets to provide feedback to students on their understanding
\end{joblong}


%----------------------------------------------------------------------------------------
%	SKILLS
%----------------------------------------------------------------------------------------
\section{Skills and Interests}
\begin{tabularx}{\linewidth}{@{}l l X@{}}
Programming Languages & Advanced & \normalsize{Python (Numpy, Pandas, Matplotlib, Astropy, Seaborn)}\\
 & Intermediate & \normalsize{SQL, Java}\\
 & Beginner & \normalsize{Matlab, Stata, R, Maple}\\
Software & \multicolumn{2}{l}{\normalsize{Tableau Desktop, Tableau Prep, Snowflake, Excel, Salesforce, }}\\
& \multicolumn{2}{l}{\normalsize{Adobe Illustrator, Adobe Photoshop}}\\
Hobbies & \multicolumn{2}{l}{\normalsize{Running, Skiing, Reading, Cooking, Art Museums, Restaurants}}\\

\end{tabularx}

\vfill
\center{\footnotesize Last updated: \today}

\end{document}
